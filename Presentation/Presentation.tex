\documentclass[french]{beamer}

\usepackage[utf8]{inputenc}
\usepackage[T1]{fontenc}
\usepackage{lmodern}
\usepackage{babel}
\usepackage{tikz}
\usepackage{graphicx}
\usetikzlibrary{arrows}

\usetheme{PaloAlto}

\tikzset{
  treenode/.style = {align=center, inner sep=0pt, text centered,
    font=\sffamily},
  arn_equi/.style = {treenode, circle, white, draw=green, fill=green, text width=1em},
  arn_notequi/.style = {treenode, circle, white, draw=red,fill=red, text width=1em},
  arn_new/.style = {treenode, circle, white, draw=blue,fill=blue, text width=1em},
  arn_null/.style = {treenode, rectangle, draw=black,minimum width=0.5em, minimum height=0.5em}
}



%Pour le TITLEPAGE
\title{Compression}
\subtitle{Projet Mathématiques et Informatique}
\author[]{LABADENS Lucas, \\ MARINO Isabelle}
\date{13 Juin 2016}
\institute[L3 S6-- Informatique]{Université Paris 7 Diderot}


\begin{document}

\begin{frame}
	\titlepage
\end{frame}

\begin{frame}
	\frametitle{Sommaire}
	\tableofcontents	
\end{frame}

\section{La compression: définition }
\begin{frame}{La Compression}
	\textbf{Plusieurs types de compression} :
	\begin{itemize}
	\item<2-3>  Compression sans perte de données
	\item<3>  Compression avec perte de données
	\end{itemize}
\end{frame}

\section{Huffman}
\begin{frame}{Huffman}
	\begin{center}
	blabla
	\end{center}
\end{frame}


\begin{frame}{Exemple}
	\begin{center}
tu peux faire 2 frame avec une pour la compression et une pour la décompression 
	\end{center}
\end{frame}

\begin{frame}{Performances}
	\begin{center}
	exemple d'inclusion de graphe 
	%\includegraphics[width=6cm]{structures.png}
	\end{center}
\end{frame}

\section{Lempel-Ziv}
\begin{frame}{Lempel-Ziv}
	\begin{center}
	blabla
	\end{center}
\end{frame}


\begin{frame}{Exemple: Compression}
	\begin{center}
	\end{center}
\end{frame}
\begin{frame}{Exemple: Décompression}
	\begin{center}
	\end{center}
\end{frame}

\begin{frame}{Performances}
	\begin{center}
	\textbf{Plusieurs suites de compression} :
	\begin{itemize}
	\item[]<2>  	blb%\includegraphics[width=6cm]{structures.png}
	\item[]<3>  	nn%\includegraphics[width=6cm]{structures.png}
	\end{itemize}
	\end{center}
\end{frame}



\section{Différences entre les algorithmes}
\begin{frame}{Différences de structures}
	\begin{center}
	%\includegraphics[width=7.9cm]{dependances.png}
	\end{center}
\end{frame}
\begin{frame}{Temps d'exécution à la compression} 
	\begin{center}
	%\includegraphics[width=7.9cm]{dependances.png}
	\end{center}
\end{frame}
\begin{frame}{Temps d'exécution à la décompression}
	\begin{center}
	%\includegraphics[width=7.9cm]{dependances.png}
	\end{center}
\end{frame}

\section{Différences avec les principaux compresseurs}
\begin{frame}{Principales différences}
	\begin{center}
	%\includegraphics[width=10cm]{fonctions.png}
	\end{center}
\end{frame}

\section{Conclusion}
\begin{frame}{Principales différences}
	\begin{center}
	\end{center}
\end{frame}

\end{document}

